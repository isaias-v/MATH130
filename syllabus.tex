% Options for packages loaded elsewhere
\PassOptionsToPackage{unicode}{hyperref}
\PassOptionsToPackage{hyphens}{url}
%
\documentclass[
  11pt,
]{article}
\usepackage{lmodern}
\usepackage{amssymb,amsmath}
\usepackage{ifxetex,ifluatex}
\ifnum 0\ifxetex 1\fi\ifluatex 1\fi=0 % if pdftex
  \usepackage[T1]{fontenc}
  \usepackage[utf8]{inputenc}
  \usepackage{textcomp} % provide euro and other symbols
\else % if luatex or xetex
  \usepackage{unicode-math}
  \defaultfontfeatures{Scale=MatchLowercase}
  \defaultfontfeatures[\rmfamily]{Ligatures=TeX,Scale=1}
\fi
% Use upquote if available, for straight quotes in verbatim environments
\IfFileExists{upquote.sty}{\usepackage{upquote}}{}
\IfFileExists{microtype.sty}{% use microtype if available
  \usepackage[]{microtype}
  \UseMicrotypeSet[protrusion]{basicmath} % disable protrusion for tt fonts
}{}
\makeatletter
\@ifundefined{KOMAClassName}{% if non-KOMA class
  \IfFileExists{parskip.sty}{%
    \usepackage{parskip}
  }{% else
    \setlength{\parindent}{0pt}
    \setlength{\parskip}{6pt plus 2pt minus 1pt}}
}{% if KOMA class
  \KOMAoptions{parskip=half}}
\makeatother
\usepackage{xcolor}
\IfFileExists{xurl.sty}{\usepackage{xurl}}{} % add URL line breaks if available
\IfFileExists{bookmark.sty}{\usepackage{bookmark}}{\usepackage{hyperref}}
\hypersetup{
  pdftitle={Math 130: Introduction to R - Fall 20},
  hidelinks,
  pdfcreator={LaTeX via pandoc}}
\urlstyle{same} % disable monospaced font for URLs
\usepackage[margin=1in]{geometry}
\usepackage{longtable,booktabs}
% Correct order of tables after \paragraph or \subparagraph
\usepackage{etoolbox}
\makeatletter
\patchcmd\longtable{\par}{\if@noskipsec\mbox{}\fi\par}{}{}
\makeatother
% Allow footnotes in longtable head/foot
\IfFileExists{footnotehyper.sty}{\usepackage{footnotehyper}}{\usepackage{footnote}}
\makesavenoteenv{longtable}
\usepackage{graphicx,grffile}
\makeatletter
\def\maxwidth{\ifdim\Gin@nat@width>\linewidth\linewidth\else\Gin@nat@width\fi}
\def\maxheight{\ifdim\Gin@nat@height>\textheight\textheight\else\Gin@nat@height\fi}
\makeatother
% Scale images if necessary, so that they will not overflow the page
% margins by default, and it is still possible to overwrite the defaults
% using explicit options in \includegraphics[width, height, ...]{}
\setkeys{Gin}{width=\maxwidth,height=\maxheight,keepaspectratio}
% Set default figure placement to htbp
\makeatletter
\def\fps@figure{htbp}
\makeatother
\setlength{\emergencystretch}{3em} % prevent overfull lines
\providecommand{\tightlist}{%
  \setlength{\itemsep}{0pt}\setlength{\parskip}{0pt}}
\setcounter{secnumdepth}{-\maxdimen} % remove section numbering
\RequirePackage{accsupp}
\RequirePackage{pdfcomment}
\newcommand{\AccTool}[2]{\BeginAccSupp{method=pdfstringdef,unicode,Alt={{#1}}}\pdftooltip{{#2}}{{#1}}\EndAccSupp{}}

\title{Math 130: Introduction to R - Fall 20}
\author{}
\date{\vspace{-2.5em}}

\begin{document}
\maketitle

\href{syllabus.pdf}{{[}Download this syllabus as a PDF{]}}

\hypertarget{course-description}{%
\section{Course Description}\label{course-description}}

This course is designed as a primer to get the complete novice up and
running with the basic knowledge of how to use the statistical
programming language R in an environment that emphasizes reproducible
research and literate programming for data analysis. The target audience
is anyone who wants to do their own data analysis. The course will
cumulate with an peer-evaluated exploratory data analysis on either a
pre-specified data set or your data set of choice.

\hypertarget{logistics}{%
\section{Logistics}\label{logistics}}

\begin{itemize}
\tightlist
\item
  \textbf{Course Website:}
  \url{https://norcalbiostat.github.io/MATH130/}
\item
  \textbf{Prerequisites:} Basic computer literacy
\item
  \textbf{Workshop style:} This workshop style class runs for 5 weeks
  only. See expectations for time commitment during these 5 weeks below.
\end{itemize}

\begin{longtable}[]{@{}lcc@{}}
\toprule
& Section 01 & Section 02\tabularnewline
\midrule
\endhead
\textbf{Meeting Days} & 8/24/20 - 9/25/20 & Not Offered\tabularnewline
\textbf{Meeting Times} & MWF 9:00-9:50 &\tabularnewline
\textbf{Meeting Location} & Online &\tabularnewline
\textbf{Instructor} & Nicholas Lytal &\tabularnewline
\textbf{Office Location} & N/A &\tabularnewline
\textbf{Office Phone} & N/A &\tabularnewline
\textbf{E-mail} &
\href{mailto:nyltal@csuchico.edu}{\nolinkurl{nyltal@csuchico.edu}}
&\tabularnewline
\bottomrule
\end{longtable}

\hypertarget{additional-support}{%
\subsection{Additional Support}\label{additional-support}}

\begin{itemize}
\tightlist
\item
  The Data Science Initiative offers year-round training workshops and
  seminars on data-science related topics including R. See
  \url{http://datascience.csuchico.edu} for more information and a
  schedule of events.
\item
  Community Coding: Times TBD. \href{cc_flyer.pdf}{{[}flyer{]}}

  \begin{itemize}
  \tightlist
  \item
    Students, staff, faculty, and the public are invited to join our
    Community Coding sessions. Bring your computer, coding projects, and
    your questions to this open working environment.
  \end{itemize}
\end{itemize}

\hypertarget{everyone-is-welcome-here}{%
\subsection{Everyone is welcome here}\label{everyone-is-welcome-here}}

It is my intent that students from all diverse backgrounds and
perspectives be well-served by this course, that students' learning
needs be addressed both in and out of class, and that the diversity that
the students bring to this class be viewed as a resource, strength and
benefit. It is my intent to present materials and activities that are
respectful of diversity: gender identity, sexuality, disability, age,
socioeconomic status, ethnicity, race, nationality, religion, and
culture. Your suggestions are encouraged and appreciated. Please let me
know ways to improve the effectiveness of the course for you personally,
or for other students or student groups.

I would like to create a learning environment that supports a diversity
of thoughts, perspectives and experiences, and honors your identities
(including race, gender, class, sexuality, religion, ability, etc.) To
help accomplish this:

\begin{itemize}
\tightlist
\item
  If you have a name and/or set of pronouns that differ from those that
  appear in your official Chico records, please let me know!
\item
  If you feel like your performance in the class is being impacted by
  your experiences outside of class, please don't hesitate to come and
  talk with me. I want to be a resource for you. Remember that you can
  also submit anonymous feedback (which will lead to me making a general
  announcement to the class, if necessary to address your concerns).
\item
  If you prefer to speak with someone outside of the course, the Office
  of Diversity and Inclusion is here to assist. Their number is
  530-898-4764, and email
  \href{mailto:diversityoffice@csuchico.edu}{\nolinkurl{diversityoffice@csuchico.edu}}
\item
  I (like many people) am still in the process of learning about diverse
  perspectives and identities. If something was said in class (by
  anyone) that made you feel uncomfortable, please talk to me about it.
  (Again, anonymous feedback is always an option).
\end{itemize}

\emph{Adapted from
\href{https://www.brown.edu/sheridan/teaching-learning-resources/inclusive-teaching/statements}{Monica
Linden at Brown University}}.

Furthermore, I would like to acknowledge that we are meeting on the
traditional lands of the Mechoopda people. Without them, we would not
have access to this campus or our education.

\hypertarget{learning-outcomes}{%
\section{Learning Outcomes}\label{learning-outcomes}}

By the end of the course, students will be able to

\begin{itemize}
\tightlist
\item
  Import data into R from external files such as text files and
  spreadsheets.
\item
  Calculate summary statistics.
\item
  Create new variables using different data types.
\item
  Perform data management techniques such as subsetting, grouping,
  summarizing.
\item
  Create informative data visualizations and tables.
\item
  Create a reproducible research document.
\item
  Conduct an exploratory data analysis in a reproducible manner.
\end{itemize}

\hypertarget{required-materials}{%
\section{Required Materials}\label{required-materials}}

\begin{itemize}
\tightlist
\item
  A reliable laptop, chromebook, tablet that can use a browser to access
  the internet.
\item
  Reliable internet connection while on and off campus.
\end{itemize}

\hypertarget{software}{%
\paragraph{Software}\label{software}}

If you have your own PC/Mac/*nix computer, you are advised to install R
and R Studio on your personal computer. That way you can take what
you've learned in this class and apply it to other classes. Both are
free. Walk through installation instructions can be found here:
\url{https://norcalbiostat.netlify.com/post/software-overview/}

\begin{itemize}
\tightlist
\item
  R version 4.0.0 or later. Download from
  \url{https://cran.r-project.org/}
\item
  R Studio version 1.39 or later. Download the desktop version from
  \url{https://www.rstudio.com/products/rstudio/download/\#download}
\end{itemize}

If you are unable to obtain a laptop that has the ability to install
program (i.e.~a iPad or Chromebook) then a cloud version of R and R
Studio is available to you at no charge. Learn more and make an account
here: \url{https://rstudio.cloud/}.

\begin{quote}
This is pending
\end{quote}

\hypertarget{time-commitment}{%
\section{Time Commitment}\label{time-commitment}}

\begin{quote}
For all CSU degree programs and courses bearing academic credit, the
``credit hour'' is defined as \ldots{} not less than one hour of
classroom or direct faculty instruction and a minimum of two hours of
out-of-class student work each week for approximately fifteen weeks for
one semester or trimester hour of credit.
\end{quote}

This adds up to 15 hours in class, and 30 hours outside of class during
these 5 weeks. That's 3 hours in class, and 10 hours of homework per
week. If you are new to programming and unfamiliar with computers, you
may end up taking the entire time. Be sure to schedule sufficient time
during week 5 to work on the project.

You will get out of this class what you put into it. Recall this is just
a co-curriculuar or supplemental basic introductory class. You will not
learn everything there is to know about R, nor likely not feel
proficient by the time you are done. But you will be solidly on the path
where you can continue to learn and improve.

\hypertarget{grading}{%
\section{Grading}\label{grading}}

Credit / No Credit. There are 100 points available in this course. You
must earn 70 points to receive credit for the course.

\begin{itemize}
\tightlist
\item
  Daily attendance: (10 pts)
\item
  Assignments:

  \begin{itemize}
  \tightlist
  \item
    HW 1 (15 pts)
  \item
    HW 2 (15 pts)
  \item
    HW 3 (15 pts)
  \item
    HW 4 (15 pts)
  \end{itemize}
\item
  Project:

  \begin{itemize}
  \tightlist
  \item
    Exploratory Data Analysis (25 pts)
  \item
    Peer Review (5 pts)
  \end{itemize}
\end{itemize}

\hypertarget{schedule-of-topics}{%
\section{Schedule of Topics}\label{schedule-of-topics}}

The general outline of topics is listed below. A detailed most up to
date schedule can be found on the course website.

\begin{itemize}
\tightlist
\item
  Week 1

  \begin{itemize}
  \tightlist
  \item
    Intro to the R language and the R Studio Integrated Development
    Environment
  \item
    Conducting reproducible research with R Markdown
  \end{itemize}
\item
  Week 2

  \begin{itemize}
  \tightlist
  \item
    How to use functions to get things done
  \item
    Introduction to data processing
  \end{itemize}
\item
  Week 3

  \begin{itemize}
  \tightlist
  \item
    Univariate Data Visualization using base, and \texttt{ggplot2}
    graphics
  \item
    Streamlined Data processing and Aggregation with \texttt{dplyr}
  \end{itemize}
\item
  Week 4

  \begin{itemize}
  \tightlist
  \item
    Bivariate and Multivariate Data Visualization using \texttt{ggplot2}
  \item
    Importing data into R from external files
  \end{itemize}
\item
  Week 5: Exploratory Data Analysis (individual project)
\end{itemize}

\begin{center}\rule{0.5\linewidth}{0.5pt}\end{center}

\hypertarget{university-policies-and-campus-resources}{%
\section{University Policies and Campus
Resources}\label{university-policies-and-campus-resources}}

\hypertarget{adding-and-dropping-the-course}{%
\subsubsection{Adding and Dropping the
course}\label{adding-and-dropping-the-course}}

This course only runs for a few weeks and all materials are available on
the course website. It will be difficult to get caught up if you add the
class after the first week. The last day to add or drop classes without
special permission by the instructor is 9/4/20. No adds or drops are
allowed after 9/18/20 without a serious and compelling reason approved
by the instructor, department chair, and college dean.

\hypertarget{academic-integrity}{%
\subsubsection{Academic Integrity}\label{academic-integrity}}

Students are expected to be familiar with the University's Academic
Integrity Policy. Your own commitment to learning, as evidenced by your
enrollment at California State University, Chico, and the University's
Academic Integrity Policy requires you to be honest in all your academic
course work. Faculty members are required to report all infractions to
the Office of Student Judicial Affairs. The policy on academic integrity
and other resources related to student conduct can be found on the
Student Judicial Affairs web site at
\url{http://www.csuchico.edu/sjd/integrity.shtml}.

\hypertarget{it-support-services}{%
\subsubsection{IT Support Services}\label{it-support-services}}

During the COVID-19 pandemic, the physical office for ITSS is closed
until further notice. However, most services are still available via
phone, online, or in a
\href{https://csuchico.zoom.us/j/5308984357?pwd=WUwwaXJQUTNPSnJqc1hTemxPY0ZiQT09}{Zoom
Lobby} during business hours. See the
\href{https://www.csuchico.edu/itss/}{ITSS webpage} for futher details.

\hypertarget{americans-with-disabilities-act}{%
\subsubsection{Americans with Disabilities
Act}\label{americans-with-disabilities-act}}

If you need course adaptations or accommodations because of a disability
or chronic illness, or if you need to make special arrangements in case
the building must be evacuated, please make an appointment with me as
soon as possible, or see me during office hours. Please also contact
Accessibility Resource Center (ARC) as they are the designated
department responsible for approving and coordinating reasonable
accommodations and services for students with disabilities. ARC will
help you understand your rights and responsibilities under the Americans
with Disabilities Act and provide you further assistance with requesting
and arranging accommodations.

Accessibility Resource Center 530-898-5959 Student Services Center 170
\href{mailto:arcdept@csuchico.edu}{\nolinkurl{arcdept@csuchico.edu}}
\url{http://www.csuchico.edu/arc}

\hypertarget{chico-state-basic-needs-project}{%
\subsubsection{Chico State Basic Needs
Project}\label{chico-state-basic-needs-project}}

The \textbf{Hungry Wildcat Food Pantry} provides supplemental food,
fresh produce, CalFresh application assistance and basic needs referral
services for students experiencing food and housing insecurity.

All students are welcomed to visit the Pantry located in the Student
Service Center 196, open Monday-Friday, 11am-4pm or call 530-898-4098.

Please visit the Chico State Basic Needs website
\url{http://www.csuchico.edu/basic-needs} for more information.

\hypertarget{confidentiality-and-mandatory-reoprting}{%
\subsection{Confidentiality and Mandatory
Reoprting}\label{confidentiality-and-mandatory-reoprting}}

As an instructor, one of my responsibilities is to help create a safe
learning environment on our campus. I also have a mandatory reporting
responsibility related to my role as a your instructor. I am required to
share information regarding sexual misconduct with the University.
Students may speak to someone confidentially by contacting the
Counseling and Wellness Center (898-6345) or Safe Place (898-3030).
Information on campus reporting obligations and other Title IX related
resources are available here: www.csuchico.edu/title-ix.

\end{document}
